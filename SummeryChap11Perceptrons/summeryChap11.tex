\documentclass{article}
\usepackage{titling}
\usepackage{fancyhdr}

\author{Wyatt Snyder}
\title{Summary of Chapter 11 of "Perceptrons An Introduction to Computational Geometry" by Marvin Minsky and Seymour Papert} 
\pagestyle{fancy}

%\fancyhf{}
%\lhead{\theauthor}
%\rhead{\thetitle}

\begin{document}
	\maketitle
	\pagenumbering{gobble}
	Chapter 11 starts off describing a learning machine and giving an example.
	Continuing in this manner it describes two different ways to create a learning machine. 
	The first way being creating a memory machine that remembers everything
	it was given, thus making it perfect on what it has seen before, the other way
	is to create a machine that detects patterns with a description. A perceptron is a
	different type of machine that lies between these two types.
	
	The perceptron being a vector in the data space, takes a figure X, with coefficients 
	representing where the data value is inthe data space, and then after summing the 
	coefficients of the perceptron multiplied by the values of the figure given, if this 
	result is positive and $X\in F^+$, where $F^+$ is the set of values in the positive subset,
	then all is well, otherwise the coefficients of the perceptron that are not zero should be 
	changed so that the value better predicts the figure.
	
	The book then continues to describe the perceptron convergence theorem, where you choose
	any value for A, the vector to represent the perceptron, and then choose any value, X, from
	either of the data sets and if the sum is positive and $X\in F^+$ or the sum is negative and
	$X\in F^-$ then continue, if not then replace A
	with $A\pm X$ to correct the sum. Then claims that the function will correct the
	vector A only a finite amount of times.
	
	Next the book refines the convergence theorem and then goes on to prove it. The proof
	starts by first defining a function $G(A)=\frac{A^* \cdot A}{|A|}$, where $A^*$ is the
	solution vector such that there exists a $\delta > 0$ such that $A^* \cdot X > \delta$, 
	and then shows how the numerator of G grows linearly with n, where n is the number of passes
	and then shows how the denominator grows slower
	than linear with respect to n. Then using the fact that G(A) is the cosine of the angle between $A^*$ and A,
	thus it has to be in between 1 and -1, the book shows that the function can only iterate a finite 
	amount of time.
	
	Continuing the book creates a geometric proof for the convergence theorem. The proof shows
	how normally the vector A will come closer to the solution vector $A^*$, with the special
	case of how it could "over shoot" going past the solution vector. But because
	$A^* \cdot X > \delta$, as the cone of the area gets larger eventually the area the vector A
	can land gets smaller than the area that would make it closer than it was, thus it will 
	eventually stop changing.
	
	The book quickly notes that the convergence theorem has a large number of variations.
	These variations are created by modifying either the set F or how the vector is changed
	during the ADD part of the algorithm. A major variation is to allow more than two classes
	of input figures, thus generalizing the theorem.
	
	Additionally the book compares the perceptron convergence theorem to the hill climbing
	problem. This can be done by defining G(A) as in the proof as to be the surface. It 
	differs from the usual hill climbing problem with how there is no systematic way to
	explore the effects of going in every direction from the current position of the
	perceptron, and how it never has the value of the function G(A) because by definition
	$A^*$ is unknown.
	
	
	 
	
\end{document}



