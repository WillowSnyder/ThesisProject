\documentclass{article}
\title{A Machine Learning Approach to Melody Identification}
\author{Wyatt Snyder}


\begin{document}
	\maketitle
	\section{Summary}
	While many attempts at song identification have been pursued and done successfully,
	there is a lack of work done in the area of melody identification, where only the melody
	line is sung or hummed to identify the musical piece. My thesis will be an investigation
	into a machine-learning centric solution for melodic identification. The research will use 
	MIDI files containing the melody lines to be analyzed and utilize Hidden Markov Models with
	varying structures to determine a list of possible melodies based off the one given.
	
	\section{Data}
	The data used for this research is going to be primarily midi files of melodies either 
	pre-existing or created during the course of the research. The use of midi files is primarily 
	because they are already encoded so they are easier to input into the learning model. Additional 
	data samples that may be used include live recordings of a melody being sung or played.
	
	\section{Features}
	The following features will be used during this thesis for the Hidden Markov Models to analyze:
	\begin{itemize}
		\item Intervals between notes: The intervals between notes is the primary feature in this 
		thesis because the intervals between tones in a melody line are what differentiate melody lines 
		from one another and define the melodic line. 
		
		\item Rhythm: The rhythm, or relative length, of the notes in the melody is another primary way 
		melodies are identified and although they are important to the identification of melodic lines
		they do not hold as much importance as the intervals do so they will not be considered with as
		much weight in this thesis project.
	\end{itemize} 
	
	\section{Learning Model}
	Hidden Markov Models will be used for this project due to the linear nature of the data being 
	analyzed and Hidden Markov Models previous use in related projects having to do with musical melody lines. 
	
	Two main different structures of Hidden Markov Models will be explored. The first structure being 
	where each melody has a single Hidden Markov Model representing it, where it can transition to any
	part of the model from the start vertex and then from any other vertex transition to a special end 
	vertex, thus allowing the melody to start and end at any part of the melody allowing for the new melody 
	to not need to be the full melody from start to end.
	
	The second main structure to be tested is each melody having several Hidden Markov Models representing 
	it with each model representing an overlapping segment of the melody. So when a melody is tested, it would depend on 
	the success of multiple Hidden Markov Models to identify the melody line.
	
	Each model has it's strengths and weaknesses the first one being better for longer continuous melodies but has the 
	possibility of incorrectly starting at a different part of the melody line and being unable to correct itself.
	While the second representation removes the problem of starting in the incorrect place it could perform worse in regards
	to longer more continuous melody line samples.
		
	\section{Key Questions}
	Key questions that will be explored during my research will include the following. How many notes are necessary 
	to identify the song? What if there are mistakes in the performance of the melody? What if there isn't an close 
	match for any melodies in that database? 
	
	\section{References}
	Cuddy, Lola L., Annabel J. Cohen, and Janet Miller. "Melody Recognition: The Experimental Application of Musical Rules." 
	Canadian Journal of Psychology, vol. 33, 1979, pp. 148. ProQuest,
	http://ezproxy.baylor.edu/login?url=https://\\search.proquest.com/docview/1289971631?accountid=7014.
	\\ \\
	“EN.MIDIMELODY.RU.” En.midimelody.ru, en.midimelody.ru/.
	\\ \\
	Johnson, Peter. “Musical Works, Musical Performances.” The Musical Times, vol. 138, no. 1854, 1997, p. 4., doi:10.2307/1003750.
	\\ \\
	Müllensiefen, Daniel, and Klaus Frieler. “Evaluating Different Approaches to Measuring the Similarity of Melodies.” 
	Studies in Classification, Data Analysis, and Knowledge Organization Data Science and Classification, pp. 299–306., doi:10.1007/3-540-34416-0\_32.
	\\ \\
	Aloupis, Greg, et al. “Algorithms for Computing Geometric Measures of Melodic Similarity.” Computer Music Journal, vol. 30, no. 3, 2006, pp. 67–76., \\doi:10.1162/comj.2006.30.3.67.
	\\ \\
	Chai, Wei, and Barry Vercoe. "Folk music classification using hidden Markov models." Proceedings of international conference on 
	artificial intelligence. Vol. 6. No. 6.4. sn, 2001.
	\\ \\
	Basili, Roberto, Alfredo Serafini, and Armando Stellato. "Classification of musical genre: a machine learning approach." ISMIR. 2004.
	\\ \\
	Cuddy, Lola L., Annabel J. Cohen, and Janet Miller. "Melody recognition: The experimental application of musical rules." Canadian 
	Journal of Experimental Psychology 33 (1979): 148.
	\\ \\
	Miotto, Riccardo, and Nicola Orio. "Automatic identification of music works through audio matching." International Conference on 
	Theory and Practice of Digital Libraries. Springer, Berlin, Heidelberg, 2007.
	\\ \\
	Ghias, Asif, et al. "Query by humming: musical information retrieval in an audio database." Proceedings of the third ACM 
	international conference on Multimedia. ACM, 1995.
	\\ \\
	Aloupis, Greg, et al. "Computing a geometric measure of the similarity between two melodies." Proceedings of the 15th 
	Canadian Conference on Computational Geometry. 2003.
	\\ \\
	Stamp, Mark. "A revealing introduction to hidden Markov models." Department of Computer Science San Jose State University (2004): 26-56.
	\\ \\
	Rabiner, Lawrence R. "A tutorial on hidden Markov models and selected applications in speech recognition." Proceedings of 
	the IEEE 77.2 (1989): 257-286.
	\\
\end{document}