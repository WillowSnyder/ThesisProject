\documentclass{article}
\title{Thesis Proposal}
\author{Wyatt Snyder}


\begin{document}
	\maketitle
	\section{Summary}
	For my thesis I will be pursuing a machine-learning centric solution that may be used to identify melodies in musical performances. Varying features of the performance will be taken into consideration with different levels of importance. The musical performance will be of only the melody with no harmonies played with it. My research will begin by analyzing the strengths and weaknesses of different machine learning models so that I may choose an appropriate model for this thesis project. 
	
	\section{Data}
	The data used for this research is going to be primarily midi files of melodies either pre-existing or created during the course of the research. Additional data samples that may be used include live recordings of a melody being sung or played.
	
	\section{Features}
	The primary feature used for training the model will be the intervals between notes played in the midi file. Additionally, the rhythm will be a primary feature used as it plays a major role in melody identification. Tempo will be another feature taken into consideration, but will be held with lesser importance as tempos can fluctuate in a melody and be performed at all different speeds. Absolute pitch will play a minor role as melodies are considered to be the "same" melody regardless of what musical key or octave they are played in.  
	
	\section{Learning Model}
	The learning model that will be used for this research project is currently unknown. The most promising choices for a learning model include the perceptron and the Hidden Markov Model.
	
	\section{Key Questions}
	Key questions that will be explored during my research will include the following. How many notes are necessary to identify the song? What if there are mistakes in the performance of the melody? What if there isn't an close match for any melodies in that database?
	
	\section{Works Cited}
	Cuddy, Lola L., Annabel J. Cohen, and Janet Miller. "Melody Recognition: The Experimental Application of Musical Rules." Canadian Journal of Psychology, vol. 33, 1979, pp. 148. ProQuest, http://ezproxy.baylor.edu/login?url=https://search.proquest.com/docview/1289971631?accountid=7014.
	
\end{document}