\documentclass{article}
\title{Thesis Proposal}
\author{Wyatt Snyder}


\begin{document}
	\maketitle
	\section{Summary}
	My thesis is going to be using a machine learning model to create a program for melody identification. 
	
	\section{Data}
	The data used for this research is going to be primarily midi files of melodies either preexisting or created during the course of the research. 
	
	\section{Features}
	The primary feature being looked for to recognize the melody will be the intervals between notes played in the midi file. Additionally the rhythm will be a primary feature looked for as it plays another major role in melody identification. Tempo will be another feature taken into consideration but will be held with lesser importance as tempos can fluctuate in a melody and be performed at all different speeds. Absolute pitch will play only a minor role as melodies are considered to be the "same" melody regardless of what musical key or octave is played in.  
	   
	
	\section{Learning Model}
	The learning model that will be used for this research project is currently unknown 
	
	\section{Key Questions}
	
	\section{Works Cited}
	Cuddy, Lola L., Annabel J. Cohen, and Janet Miller. "Melody Recognition: The Experimental Application of Musical Rules." Canadian Journal of Psychology, vol. 33, 1979, pp. 148. ProQuest, http://ezproxy.baylor.edu/login?url=https://search.proquest.com/docview/1289971631?accountid=7014.
	
\end{document}